%!Mode:: "TeX:UTF-8"
%!TEX program = xelatex
%!TEX TS-program = xelatex
%!TEX encoding = UTF-8 Unicode
%
% Author: Rickjin (ZhihuiJin@gmail.com)
%

\chapter{和e有关的公式附录}

\section{basic}

$$  e \hspace{0.05cm} \& \hspace{0.05cm}  \pi $$

$$ e \approx  2.718281828 $$

$$ \log_ex = \ln x  $$

$$ n! \approx \sqrt{2\pi n}\big({n \over e}\big)^n $$

$$ e = \sum_{n=0}^\infty {1\over n!} = {1\over 0!} + {1 \over 1!} + {1\over 2!} + {1 \over 3!} + {1 \over 4!} + \cdots $$

$$ e^x = \sum_{n=0}^\infty {x\over n!} = 1 + {x \over 1!} + {x^2\over 2!} + {x^3 \over 3!} + {x^4 \over 4!}  + \cdots $$

$$ e^x = 1 + {x \over 1!} + {x^2\over 2!} + {x ^3\over 3!} + {x ^4\over 4!}  + \cdots $$

$$  y=e^x $$  

$$  x= \log_e y = \ln y $$

$$ y = 1 + {x \over 1!} + {x^2\over 2!} + {x ^3\over 3!} + {x ^4\over 4!}  + \cdots $$



$$ {d \over dx} \log_e x = {1 \over x} $$

$$ ( \log_e x)'  = {1 \over x} $$

$$ {d \over dx} e^x = e^x $$

$$  (e^x)' = e^x $$

$$ e = \lim_{m\rightarrow \infty} \big(1+\frac{1}{m}\big)^m $$

$$ e = \lim_{n\rightarrow \infty} \big(1+\frac{1}{n}\big)^n $$


$$ e = \lim_{n\rightarrow +\infty} \big(1+\frac{1}{n}\big)^n $$

$$ e = \lim_{n\rightarrow -\infty} \big(1+\frac{1}{n}\big)^n $$


$$ e = \lim_{x \rightarrow \infty} \big(1+\frac{1}{x}\big)^x $$

$$ e =  \big(1+\frac{1}{+\infty }\big)^{+\infty} $$

$$ e =  \big(1+\frac{1}{-\infty }\big)^{-\infty} $$


$$ e^x = \lim_{n\rightarrow \infty} \big(1+\frac{1}{n}\big)^{nx} $$


$$ e^x = \lim_{n\rightarrow \infty} \big(1+\frac{x}{n}\big)^n $$

$$ e^{i\theta}  = \cos \theta + \sin \theta $$

$$ e^{i\pi} + 1 = 0 $$

$$  \int_1^e {1\over x} dx = 1 $$

$$  \phi(x) = {1 \over \sqrt{2\pi}} e^{-\frac{x^2}{2}} $$

$$ \lim_{n \rightarrow \infty} ( 1  + {1\over2} + {1\over3} + \cdots + {1\over n} - \log_e n ) =  \gamma  $$

$$ {1\over e} = {1 \over 0!}  - {1 \over 1!} + {1\over 2!} - {1\over 3!}  + {1\over 4!} -  {1\over 5!} + \cdots $$

$$ e^2 = 1 + {2 \over 1!} + {2^2\over 2!} + {2^3\over 3!} + {2^4\over 4!}  + \cdots $$

$$ e = \lim_{n \rightarrow \infty} \frac{n}{\sqrt[n]{1  \times 2  \times 3 \cdots \times n}} $$


$$ e = 2^{  \frac{1}{ 1- {1\over2} + {1\over3} - {1\over4}  + {1\over 5} - \cdots }  }  $$

$$ e = \sqrt[(1- {1\over2} + {1\over3} - {1\over4}  + {1\over 5} - \cdots) ]{2}  $$


\section{second}


$$  \frac{e+e^{-1}}{2} = (1+\frac{1}{1^2})(1+\frac{1}{3^2})(1+\frac{1}{5^2})(1+\frac{1}{7^2}) \cdots  $$

$$  \frac{e-e^{-1}}{2} = (1+\frac{1}{1^2})(1+\frac{1}{2^2})(1+\frac{1}{3^2})(1+\frac{1}{4^2}) \cdots  $$

$$ \lim_{n\rightarrow +\infty} (1-\frac{1}{n})^n  =  \frac{1}{e}  $$  

\begin{equation}
     e = 2+ \cfrac{1}{ 1 +
     \cfrac{1}{ 2 +
     \cfrac{2}{ 3 +
     \cfrac{3}{ 4 +
     \cfrac{4}{ 5 + \dotsb}}}}}
\end{equation}


$$  \sqrt{ {1\over2} e \pi}  = 1+{1 \over 1\times 3}  +{1 \over 1\times 3 \times 5} +  {1 \over 1\times 3 \times 5 \times 7} + \cdots +
\cfrac{1}{ 1 +
  \cfrac{1}{ 1 +
    \cfrac{2}{ 1 +
      \cfrac{3}{ 1 +
        \cfrac{4}{ 1 + \dotsb}}}}}
$$

\begin{align*}
& \log { 23.712  \times   \sqrt{81234}   \times  678920.35  \times  \sqrt[3]{974372}   
 \over {  \sqrt{1376}  \times   \sqrt[3]{123455667}  }}  \\
= & \log 23.712  +   \frac1 2 \cdot \log 81234   +  \log 678920.35  +  \frac 1 3 \cdot \log{974372} \\
& -  \frac 1 2 \cdot \log{1376}  -  \frac 1 3 \cdot \log {123455667} 
\end{align*}  

$$  \log81234  =  \log(8.1234 \times 10^4)  =   \log8.1234 + 4 \log10 $$

$$ \log_{10}(x) = \lg (x) $$ 

\begin{align*}
 & (1+\frac 1 n ) ^n  \\
 =  & \binom{n}{0}  + \binom{n}{1}  \frac 1 n + \binom{n}{2} (\frac 1 n)^2 + \binom{n}{3} (\frac 1 n)^3 + \cdots  \\
 =  & 1 + n \frac 1 n + \frac {n(n-1)}{2!}  (\frac 1 n)^2  +  \frac {n(n-1)(n-2)}{3!}  (\frac 1 n)^3 + \cdots  \\
 \rightarrow  & 1 + \frac{1}{1!}  + \frac{1}{2!}   + \frac{1}{3!}  + \cdots
 \end{align*}  


$$  +\infty   \rightarrow   -\infty  $$ 

$$  \lim_{n\rightarrow + \infty } (1-\frac 1 n ) ^{-n}    =  ?  $$

$$  \lim_{n\rightarrow + \infty } (1-\frac 1 n ) ^{-n}    =  e  $$


$$  \lim_{n\rightarrow - \infty } (1+ \frac 1 n ) ^{n}    =  ?  $$



\begin{align*}
 & (1-\frac 1 n ) ^n  \\
 =  & \binom{n}{0}  - \binom{n}{1}  \frac 1 n + \binom{n}{2} (\frac 1 n)^2 - \binom{n}{3} (\frac 1 n)^3 + \cdots  \\
 =  & 1 - n \frac 1 n + \frac {n(n-1)}{2!}  (\frac 1 n)^2  -  \frac {n(n-1)(n-2)}{3!}  (\frac 1 n)^3 + \cdots  \\
 \rightarrow  & 1 -\frac{1}{1!}  + \frac{1}{2!}   - \frac{1}{3!}  + \cdots
 \end{align*}  


\begin{align*}
& x  \times   y  = ? \hspace{0.3cm}  x  \div  y  = ? \\
& x  =  a^m ,     y =  a^n   \\
& x  \times  y  =  a^{m}   \times  a^{n}  = a^{m+n} = z
\end{align*}

$$  a^5 \times a^3 = a^{5+3} = a^8  $$   
$$   a^5  \div a^3 = a^{5-3} = a^{-2} $$
$$ a^{1\over2} + a^{1\over 3} = a^{\frac 1 2 + \frac 1 3} $$
$$  a^{0.25}  +  a^{1.3}  =   a^{0.25 + 1.3}  $$

\begin{tabular}{c|ccccccc}
\hline
$k$  & $\cdots$ & $m$ & $\cdots$ & $n$ & $\cdots$ & $m+n$ & $\cdots$ \\
\hline
$10^k$  & $\cdots$ &  $x$  & $\cdots$ & $y$ & $\cdots$ & $z$ & $\cdots$ \\
\hline
\end{tabular}


\begin{tabular}{c|ccccccc}
\hline
$k$  & $\cdots$ & $m$ & $\cdots$ & $n$ & $\cdots$ & $m+n$ & $\cdots$ \\
\hline
$a^k$  & $\cdots$ &  $x$  & $\cdots$ & $y$ & $\cdots$ & $z$ & $\cdots$ \\
\hline
\end{tabular}

\begin{align*}
 x & = 10^7(1- 10^{-7})^k   \\
   &=  10^7(1- 10^{-7})^{10^7 \cdot \frac k {10^7} } \\
   & \approx  10^7 \cdot  \left(\frac{1}{e}\right)^\frac{k}{10^7}  \\
   &  = \displaystyle 10^7 \cdot  e^{-\frac{k}{10^7} }
\end{align*}


$$ y = - \frac{1}{x} $$

\section{ e is irrational }


$$ e = \frac m n $$
$$ ne =  m $$ 
$$ n! \cdot ne = n! \cdot m $$
 
 \begin{align*}
 e & =  \left( 1 + {1 \over 1!} + {1\over 2!} + \cdots + {1 \over n!}   \right)   +  \left[ {1 \over (n+1)! }  + {1 \over (n+2)!}  +  {1 \over (n+3)! } + \cdots  \right] \\
 n! \cdot ne & =   n n! \left( 1 + {1 \over 1!} + {1\over 2!}  + \cdots + {1 \over n!}   \right)  \\
 &   + n\left[ {1 \over (n+1)}  + {1 \over (n+1)(n+2)}  + {1 \over (n+1)(n+2)(n+3)} + \cdots \right]
 \end {align*}
 
 
 $$ \pi \approx  \frac{2n}{m}  $$
 
 $$  \pi \approx   \frac {2 \times 17}{11} = 3.1 $$  
 
 $$ \pi \approx  \sqrt {6 \times \frac{250}{154}}  =  3.12  $$
 
$$ x \cdot y = \frac{(x+y)^2 - (x-y)^2}{4} $$

$$ \sin \alpha \cdot \sin \beta = \frac{ \cos(\alpha-\beta) - \cos(\alpha + \beta)}{2} $$ 

$$  f(x)' = f(x)    \hspace{0.2cm} ?$$

$$ y = \frac{\ln (\frac{x}{m} - sa)}{r^2} $$

