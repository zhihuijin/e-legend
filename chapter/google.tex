%!Mode:: "TeX:UTF-8"
%!TEX program = xelatex
%!TEX TS-program = xelatex
%!TEX encoding = UTF-8 Unicode
%
% Author: Rickjin (ZhihuiJin@gmail.com)
% Author: Leijun 

\chapter{从Google 说起}

\section{字母的江湖地位}
对各类事物论资排辈理出个江湖地位是人们闲时的重要消遣,世界大学的排名、科学家的
贡献、金庸小说中各位大侠的战斗力,无不是互联网网友们唾沫横飞争个面红耳赤的重要
话题。 今天我们想争论的一个问题是:在文字的世界里,哪个字母的江湖地位排名第
一?如果限制在英语的26个字母中, 这个问题也许相对容易回答,很多人应该会选择字母
$e$。即便在全世界的文字中来投票,预计字母$e$占据头把交椅的可能性也最高,毕竟英
语是准国际语言。 学过英语的读者利用自己的直觉也不难感知到字母 $e$ 是英语中使用
频率最高字母。而这种主观感觉和实际的统计结果是非常符合的,下表列出了26个英语字
母的使用频率( 这个表的来源最好说明一下)。

\begin{table}[htbp]
\centering
\caption{英文字母使用频率(百分比)}
\begin{tabular}{|l|}
\hline
A 8.19 B 1.47 C 3.83 D 3.91 E 12.25      \\ \hline
F 2.26 G 1.71 H 4.57 I 7.10 J 0.14       \\ \hline
K 0.41 L 3.77 M 3.34 N 7.06 O 7.26       \\ \hline
P 2.89 Q 0.09 R 6.85 S 6.36 T 9.41       \\ \hline
U 2.58 V 1.09 W 1.59 X 0.21Y 1.58 Z 0.08 \\ \hline
\end{tabular}
\centering
\end{table}

顺着这个思路我们抛出一个问题:如果限制在数学王国中,哪个字母能夺得桂冠呢? 虑到
希腊字母在数学世界中的霸主地位,$\pi$ 预计能够傲视其他一切数学字母。但是如果把
问题稍微改变一点:举办一场数学王国中的字母选美比赛,数学王国中的字母都可以报名
参加,哪位将夺得桂冠?这个问题可能会变得很有争议,而字母 $e$ 绝对是一个实力雄厚
的竞争者。我们的《传奇e事》系列就是要来讲讲和字母 $e$ 相关的数学故事,相信读者
们读完很多和$e$ 相关的故事之后,会更加认同 $e$ 是在数学王国中是是有绝对的竞选实
力的。 

\section{巨头的数学基因}
$e$ 在数学王国中自然是一个有故事的字母,然而我们不用着急了解它的历史, 先把时间
拉到现代,看看现代人是如何看待$e$的。 我们的故事先从著名的互联网巨头 Google 说
起。

Google 是一个大家都很熟悉的公司,她创造了世界上最庞大的搜索引擎,以免费的形式为
全世界人民提供了大量的互联网产品和服务。如果把互联网比作江湖,那么Google、
Amazon、 Facebook、Yahoo 这些公司都是来自美国的江湖大侠。同样地 BAT(百度、阿里
和腾讯)这三家公司是来自中国的大侠。那谁会是这个互联网江湖的武林盟主呢?绝大多
数人会推举 Google 作为互联网江湖的武林盟主。天下武功出少林,互联网的一流技术大
都源自 Google。Google 成为了计算机工程师(俗称码农)朝圣的地方,很多计算机系的
毕业生在毕业时都希望能进入 Google,只要在 Google 学了一招半式,脑门上就会有技术
的光环,出来以后在互联网江湖里就会变得受人尊重。

% ref http://graphics.stanford.edu/~dk/google_name_origin.html
% https://www.google.com.hk/intl/zh-CN/about/company/
然而很多人也许并不知道, Google 是一家非常具有数学基因的公司,Google 在很多的活
动和事件中都表达出她对数学的热爱,她其实是一个数学的超级大粉丝。怎么看出来的呢
?首先我们来看看 Google 这个公司的命名, "Google" 这个单词原本来自一个数学名词,
它原始的含义表示1后面跟着100个零(即10的100次方)。而这个词的原始版本 Googol
可是由一个9岁大的小屁孩创造的,这个小屁孩就是美国数学家Edward Kasner 的外甥
Milton Sirotta, 而 Googol 这个词通过Kasner 和 James Newman 合著的《数学和想象
》( Mathematics and the Imagination)一书而广为流传。

既然最早的词是 Googol, 为何最终公司名字会演变成 Google 了呢?这背后有一些有意思
的故事。 Google 的创始人 Larry Page 和 Sergey Brin 梦想整合互联网上海量信息,在
1997年的时候他们考虑为公司注册一个和海量数据相关的域名。两位创始人和几位大四本
科生一起头脑风暴,在本科生 Sean Anderson 建议下大家聚焦到了 Googol 这个名字:由
于 Googol 代表了 $10^100$, 所以毫无疑问这是一个好名字。 结果在域名服务上注册的
时候,Sean 把域名阴错阳差的敲成了"google.com" ,而实际上纠正拼写后却发现
"googol.com" 已经被别人注册了,这可如何是好? 然而 Larry 却很喜欢这个拼写错误的
的版本 "google", 于是他们迅速敲定为公司注册了域名 "google.com" 。 


第二个反映 Google 数学基因的例子就是Google 涂鸦(Google Doodle),Google 创造了一
种艺术形式,会在一些重要的日子把自己门面上的 Logo 通过各种艺术形式换成经过专门
设计的图形,用来纪念跟这个重要日子相关的人物或事件;而在和数学相关的重要节日中
, Google 就会推出一些别出心裁的数学涂鸦。下图种列举了一些比较有名的数学涂鸦,
你能猜出来每一个涂鸦都是在纪念哪些数学的事件或者人物吗?
\footnote{
从左到右,从上到下分别是:
2014年3月14日\href{http://www.google.com/doodles/pi-day?hl=zh-CN}{圆周率日},
2010年10月11日\href{http://www.google.com/doodles/cahit-arfs-100th-birthday?hl=zh-CN}{贾希特·阿尔夫诞辰100周年} ,
2004年2月2日\href{http://www.google.com/doodles/gaston-julias-111th-birthday?hl=zh-CN}{加斯顿·朱丽亚诞辰111周年},
2009年4月20日\href{http://www.google.com/doodles/zu-chongzhis-birthday?hl=zh-CN}{祖冲之诞辰纪念日},
2011年11月12日\href{http://www.google.com/doodles/hua-luogengs-101st-birthday?hl=zh-CN}{华罗庚诞辰101周年},
2015年3月23日\href{http://www.google.com/doodles/emmy-noethers-133rd-birthday?hl=zh-CN}{埃米·诺特诞辰 133 周年},
2011年8月17日\href{http://www.google.com/doodles/pierre-de-fermats-410th-birthday?hl=zh-CN}{皮埃尔·德·费玛诞辰410周年},
2013年4月15日\href{http://www.google.com/doodles/leonhard-eulers-306th-birthday?hl=zh-CN}{莱昂哈德·欧拉诞辰306周年},
2012年12月22日\href{http://www.google.com/doodles/srinivasa-ramanujans-125th-birthday?hl=zh-CN}{斯里尼瓦瑟·拉马努金诞辰125周年},
2014年8月4日\href{http://www.google.com/doodles/john-venns-180th-birthday?hl=zh-CN}{约翰·维恩诞辰180周年},
2012年6月23日\href{http://www.google.com/doodles/alan-turings-100th-birthday?hl=zh-CN}{阿兰·图灵诞辰100周年}。
}

\begin{figure}[htbp]
\centering
\includegraphics[scale=0.5]{google/doodle.png}
\caption{Google 数学涂鸦}
\centering
\end{figure}

\section{$e$之恋}

Google 2014年8月19日在纳斯达克 IPO, 提交的 IPO S-1 表格上写的金额是
2,718,281,828美元,这个数字一给出,华尔街一片哗然,不知道 Google 为什么选择这样
一个奇怪的数字,而且有零有整,精确到一美元,直接宣布融资27.1亿美元不就行了。一
般公司上市的时候会选择的融资额会精确到千万、百万美元等,几乎没有精确到一美元的
。但是数学的粉丝一看到这个数字,非常的兴奋,他们对这个数字再熟悉不过了。这个数
就是无理数 $e$ 的前十位。上市对一个公司来说可以算最重要的事件,Google 选择在上
市的时候选择这样一个融资额度,充分表达了 Google 对 $e$ 的喜爱和致敬。

大家都知道,数学中有三个非常著名的无理数, 圆周率 $\pi$、自然常数 $e$ 和 黄金分
割比 $\phi$ , Google 对这三大无理数都特别喜欢。Google 第二幢办公楼的名称就叫
$e$,第三幢办公楼叫 $\pi$,第四幢楼则命名为 $\phi$。
\begin{figure}[htbp]
\centering
\includegraphics[scale=0.5]{google/dice.png}
\caption{三大无理数}
\centering
\end{figure}

Google 对数学的喜爱无处不在。2004年7月在美国加州硅谷101公路的旁边出现了一个大大
的广告牌,广告牌特别奇怪,只有一行字,"$\{$the first 10-digit prime in
consecutive digits of $e$ $\}$.com",如图三所示。很快,类似的广告牌纷纷出现在了
西雅图、哥伦比亚、马塞诸塞、华盛顿、奥斯丁、德克萨斯等美国各大州,引起了极大的
关注度。很多人不禁感到奇怪,哪个公司这么有钱,打这么一个谁也看不懂的奇怪广告。
数学粉丝们一看就知道这是一个数学题,常数 $e$ 中出现的第一个10位质数,很多数学粉
丝纷纷开始了挑战。数学爱好者们写完程序,算出来这个数字是7427466391,然后登录
7427466391.com,发现网站里面又出现了一道新的数学题,比广告牌里的数学题还难。数
学爱好者们又被这道题所吸引,攻克完这道题之后,接下来又遇到了几道题,经过层层攻
克之后,数学爱好者们最终看到了 Google Lab 的一个招聘页面,大家终于明白原来那个
奇怪的广告牌是 Google 公司的一个招聘广告。Google 的工程师们选择了 $e$ 这样一个
无理数作为面试题,也充分表达了对无理数 $e$ 的喜爱。
\begin{figure}[htbp]
\centering
\includegraphics[scale=0.8]{google/billboard.jpg}
\caption{Google 招聘广告}
\centering
\end{figure}
	
\section{$e$ 是什么}
讲完了 Google 和 $e$ 的故事,最后来看看 $e$ 到底是什么?中学里我们就接触过 $e$
,那时候我们知道 $e$ 约等于 2.718281828,$e$ 是自然对数的底数,$e$ 的定义是一个
极限。
\begin{equation}
\nonumber
\begin{split}
e \approx 2.718281828 \\
ln(x) = log_{e}(x) \\
e = \lim_{n \to \infty}(1+\frac{1}{n})^n
\end{split}
\end{equation}
在上大学学习微积分、概率论、数学分析以后就会接触到越来越多跟 $e$ 相关的知识。
$e$ 到底是一个什么样的数? $e$ 有什么有趣的性质?$e$ 和我们的日常生活有什么联系
?历史上数学家是如何发现 $e$ 的?$e$ 有多少年的历史? 为何以 $e$ 为底的对数称为
自然对数? 为什么 $e$ 有那么多名字?自然常数、欧拉常数、纳皮尔常数。为什么
Google 和许多数学人如此喜欢 $e$?

最后放上一张由字母 $e$ 组成的单词图片,这个图片里面的每一个单词背后都跟 $e$ 的
某些性质相关。著名的数学科普大师马丁·加德纳(Martin Gardner,1914年10月21日—
2010年05月22日)发现,三个无理数——圆周率 $\pi$、自然常数 $e$ 和 黄金分割比
$\phi$——中,学生们对自然常数 $e$ 是最不熟悉的,如果读者也不熟悉的话,欢迎接下
来的时间一起学习关于 $e$ 的传奇故事。
\begin{figure}[htbp]
\centering
\includegraphics[scale=0.5]{google/eword.png}
\caption{$e$ 单词云}
\centering
\end{figure}


